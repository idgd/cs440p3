\documentclass[11pt]{article}

\usepackage{setspace}
\usepackage[activate={true,nocompatibility},
            final,
            tracking=true,
            kerning=true,
            spacing=true,
            factor=1100,
            stretch=10,
            shrink=10]{microtype}
\usepackage{listings}
\usepackage{xcolor}

\microtypecontext{spacing=nonfrench}

\doublespacing

\setlength\parskip{1ex}
\setlength\parindent{1em}

\begin{document}

\begin{enumerate}

\item Use the following code to illustrate the Banker’s algorithm and explain what is occurring at each step. Annotate your output to illustrate what is happening.

\end{enumerate}

\begin{center}
\lstinputlisting[language=Java,
                 basicstyle=\footnotesize \ttfamily,
                 commentstyle=\color{green},
                 keywordstyle=\color{blue},
                 stringstyle=\color{purple},
                 showstringspaces=false,
                 numbers=left,
                 numbersep=5pt,
                 numberstyle=\tiny,
                 title=\lstname]{Bankers.java}

\newpage

\begin{lstlisting}[basicstyle=\footnotesize \ttfamily,
       numbers=left,
       numbersep=5pt,
       numberstyle=\tiny,
       title=Output]
////SUCCESSFUL

Enter no. of processes and resources : 2 1 //2 processes, 1 resource
Enter allocation matrix -->
2 2 //already has 2 allocated
Enter max matrix -->
4 4 //can go up to 4
Enter available matrix -->
6 6 //there's 6 available for both processes
Allocated process : 0
Allocated process : 1
Safely allocated //(4 - 2) < 6, you're good to go


////UNSUCCESSFUL

Enter no. of processes and resources : 2 1 //2 processes, 1 resource
Enter allocation matrix -->
2 2 //already has 2 allocated
Enter max matrix -->
4 4 //can go up to 4
Enter available matrix -->
1 1 //only has 1 available for both
All proceess cant be allocated safely //(4 - 2) > 2, fails

\end{lstlisting}

\end{center}

\newpage

\begin{enumerate}
\setcounter{enumi}{1}

\item Give examples of inputs where a safe allocation of processes occurs and one where processes cannot be allocated safely.

\textbf{See Output, above. If \texttt{(allocation - max) \textgreater available}, allocation fails.}

\item What conditions cause the former to happen? The latter? Clearly indicate these in your writeup. (e.g., for all i, j, when max[i][j] \textless avail[i][j])

\textbf{See answer 2. Output contains both a failure and success and explains why; resources available were less than resources needed by a process.}

\item From a big picture perspective, why is this implementation of resource allocation so widely appreciated?

\textbf{It's so simple that it can be understood and implemented intuitively, but it solves a majority of the problems associated with resource allocation. It guarantees that resources are not exhausted, and that deadlock never occurs; it can be used to help schedulers work properly, and it prevents the deadliest errors in resource allocation, all in a function which only takes 70 ish lines of Java to implement. Its elegance and guarantees are what makes it so famous.}

\end{enumerate}

\end{document}

\documentclass[11pt]{article}

\usepackage{setspace}
\usepackage[activate={true,nocompatibility},
            final,
            tracking=true,
            kerning=true,
            spacing=true,
            factor=1100,
            stretch=10,
            shrink=10]{microtype}
\usepackage{listings}
\usepackage{xcolor}

\microtypecontext{spacing=nonfrench}

\doublespacing

\setlength\parskip{1ex}
\setlength\parindent{1em}

\begin{document}

\begin{enumerate}

\item Use the following code to illustrate the Banker’s algorithm and explain what is occurring at each step. Annotate your output to illustrate what is happening.

\end{enumerate}

\begin{center}
\lstinputlisting[language=Java,
                 basicstyle=\footnotesize \ttfamily,
                 commentstyle=\color{green},
                 keywordstyle=\color{blue},
                 stringstyle=\color{purple},
                 linewidth=\textwidth,
                 showstringspaces=false,
                 numbers=left,
                 numbersep=5pt,
                 numberstyle=\tiny]{Bankers.java}
\end{center}

\begin{enumerate}
\setcounter{enumi}{1}

\item Give examples of inputs where a safe allocation of processes occurs and one where processes cannot be allocated safely.

\item What conditions cause the former to happen? The latter? Clearly indicate these in your writeup. (e.g., for all i, j, when max[i][j] \textless avail[i][j])

\item From a big picture perspective, why is this implementation of resource allocation so widely appreciated?

\end{enumerate}

\end{document}
